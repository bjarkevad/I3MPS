\documentclass{paper}
\usepackage[margin=0.5in]{geometry}
\usepackage{graphicx}
\usepackage{listings}

% \includegraphics[scale=1]{img.png}
% \begin{lstlisting} 
\title{IOCTL + Kernel Timers}
\begin{document}
\maketitle
\line(1,0){510}
\begin{itemize}
\item \textit{Hvad benytter man IOCTL kald til?}
\item \textit{Hvordan implementeres IOCTL funktioner i en device driver og hvordan tilg\aa s disse fra user-space?}
\item \textit{Hvordan benytter man timers / delays i kernen?}
\item \textit{Hvilke fordele / ulemper har de forskellige delay / timer typer?\\}
\end{itemize}
\begin{large}\textbf{Hvad benytter man IOCTL kald til?}\end{large}\\
\line(1,0){510}
\begin{itemize}
\item Direkte systemkald
\item Tilg\aa\ modul fra userspace
\item Ops\ae tning af modul fra userspace
\item IOCTL kald er globale i kernen
\item `Input/Output control' \\
\end{itemize}

\begin{large}\textbf{Hvordan implementeres IOCTL funtioner i en device driver og hvordan tilg\aa s disse fra user-space?}\end{large}\\
\line(1,0){510}
\begin{itemize}
	\item F\o rst skal man finde et ledigt IOCTL nummer(Sl\aa es op i ioctl-number.txt)
	\item Der oprettes en IOCTL metode
\begin{lstlisting}
	long my_ioctl(struct file *fp, unsigned int cmd, unsigned long arg)
\end{lstlisting}
	\item Metoden registreres i fops structen
\begin{lstlisting}
.unlocked_ioctl = my_ioctl
\end{lstlisting}
	\item IOCTL kommandoer oprettes, og kan defineres som macroer:
\begin{lstlisting}
	#define START _IO(MY_IOCTL_NO, 1)
......
\end{lstlisting}
 \item IOCTL metoden implementeres til at kunne g\o re forskellige ting p\aa\ IOCTL kommandoerne
 \begin{itemize}
 \item Switch/case der kalder forskellige metoder, s\ae tter variabler osv.
 \end{itemize}
 \item IOCTL kald kan nu benyttes fra userspace:
	\begin{itemize}
		\item Userspace applikationen skal kende til IOCTL nummeret
		\item En file descriptor \aa bnes til device node
	\begin{lstlisting}
		int ioctl(int d, int request, void *..);
	\end{lstlisting}
	\end{itemize}
\end{itemize}

\begin{large}\textbf{Hvordan benytter man timers / delays i kernen?}\end{large}\\
\line(1,0){510}
\begin{itemize}
\item Den relative tid kan f\aa s fra jiffies i Linux
	\begin{itemize}
	\item jiffies er en 64 bit v\ae rdi
	\item Genstartes ved system start
	\item 'flips' efter $2^{64}$
	\end{itemize}
\item \textbf{Busy waiting}
	\begin{itemize}
	\item En 'tom' while l\o kke
	\end{itemize}
\item \textbf{Wait interruptible, wait interruptible timeout}
\item \textbf{Scheduler timeout}
\item \textbf{Sleep}

\item Kernel timere er baseret p\aa\ software interrupts
\item Timer funktioner k\o res i interrupt context
	\begin{itemize}
	\item Ingen sleeps, allokering, osv..\\
	\end{itemize}
\end{itemize}


\begin{large}\textbf{Hvilke fordele / ulemper har de forskellige delay / timer typer?}\end{large}\\
\line(1,0){510}
\begin{itemize}
\item \textbf{Busy waiting:}
	\begin{itemize}
	\item Holder p\aa\ sin CPU tid, alts\aa\ processen bliver ikke flyttet ud af schedulerens wait-queue
	\item Bruger mange CPU resourcer
	\end{itemize}
\item \textbf{Wait interruptible}
	\begin{itemize}
		\item V\aa nger n\aa r der er brug for den
		\item Sleeping
	\end{itemize}
\item \textbf{Kernel Timers}
	\begin{itemize}
	\item Opf\o rer sig som et planlagt interrupt
	\item Skal f\o lge samme regler som en ISR
	\item Ikke i process kontekst
	\end{itemize}
\end{itemize}

\end{document}

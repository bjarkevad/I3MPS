\documentclass{paper}
\usepackage[margin=0.5in]{geometry}
\usepackage{graphicx}
\usepackage{listings}

\title{4. Linux Character Drivers}
\begin{document}
\maketitle
\begin{large}\textbf{Hvad er en character driver?}\end{large}

\line(1,0){510}
\begin{itemize}
	\item Et kerne modul der h\aa ndterer skrivning og l\ae sning fra en character device
	\begin{itemize}
		\item En character device er et device der modtager data i en str\o m
		\item F.eks. en seriel port, netv\ae rks kommunikation, SPI device
		\item \textbf{Ikke} HDD!
	\end{itemize}
	\item Bliver tilg\aa et igennem en \textbf{device node}, som regel i /dev/
	\\
\end{itemize}

\begin{large}\textbf{Forklar hvad major-/minor numre er og hvordan de allokeres}\end{large}

\line(1,0){510}
\begin{itemize}
	\item Angiver hvilket kernemodul der skal benyttes
	\begin{itemize}
		\item En form for adressering af moduler fra nodes
	\end{itemize}
	\item En character driver har typisk kun et major nummer
	\begin{itemize}
		\item Minor numre benyttes s\aa\  til at tilg\aa  et "subdevice"
		\begin{itemize}
			\item F.eks. en ADC med flere kanaler kan have minor no. 0 - N-1, hvor N er antallet af kanaler
		\end{itemize}
	\end{itemize}
	\item Allokeres ved indsaettelse af kernemodul, og dern\ae st enten statisk eller dynamisk:
	\item Statisk:
	\begin{itemize}
		\item Et modul har et bestemt major no.
		\item Ikke egnet til systemer hvor man ikke har styr p\aa\  hvilke moduler bliver indsat
		\item Egnet til sm\aa\  embeddede systemer
		\\
	\end{itemize}
\end{itemize}


\begin{large}\textbf{Hvordan registrerer man et device og hvad sker der n\aa r man g\o r det?}\end{large}

\line(1,0){510}
\begin{itemize}
	\item F\o rst skal kernemodulet inds\ae ttes
	\item Dern\ae st skal der oprettes et device node\\
		\begin{lstlisting}
			mknod /dev/mychardev -c 62 0
		\end{lstlisting}
	\item Der laves et node i /dev/ med navnet mychardev, der angives det er en char device med major no. 62, og minor no. 0
	\item Herefter kan device driveren tilg\aa s fra userspace applikationer som en fil
	\\
\end{itemize}


\begin{large}\textbf{Forklar form\aa let med metoderne Open / Close / Read / Write}\end{large}

\line(1,0){510}
\begin{itemize}
	\item Metoderne benyttes til at bestemme hvad der sker, n\aa r man benytter forskellige kaldt p\aa\ character devicets device node
	\begin{itemize}
		\item N\aa r noden \aa bnes, vil open metoden blive kaldt
		\item N\aa r der skrives eller l\ae ses fra noden bliver read/write kaldt
		\item N\aa r noden lukkes, vil close metoden blive kaldt
	\end{itemize}
	\item Disse metoder kaldes \textbf{Fil Operationer}, alts\aa\ de operationer der udf\o res n\aa r devicet benyttes som en fil
	\begin{itemize}
		\item Disse operationer skal erkl\ae res i \textbf{FOPS Structen}, s\aa ledes driveren ved hvad der skal kaldes, og hvorn\aa r\\
	\end{itemize}
\end{itemize}


\begin{large}\textbf{Forklar hvordan data overf\o res mellem user- / kernel space}\end{large}

\line(1,0){510}
\begin{itemize}
	\item 
	\begin{lstlisting}
	unsigned long copy_to_user(void __user *to, const void *from, unsigned long n); 
	\end{lstlisting}
	\begin{itemize}
		\item Copies a block of data, with a specified size, to userspace
		\item 'from' is the source address in kernelspace
		\item 'to' is the destination address in userspace
		\item 'n' is the size of the block in bytes
		\item returns the amount of bytes that could \textbf{not} be copied. 0 on success
	\end{itemize}
	\item read og write metoder tager en char* ind som argument, som kan benytes til at kopiere til og fra userspace
	\item Typisk i read metoden, vil der blive kopieret data til userspace
	\item I write vil der typisk blive kopieret fra userspace
\end{itemize}
\end{document}

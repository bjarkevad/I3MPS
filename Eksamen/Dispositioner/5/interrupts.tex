\documentclass{paper}
\usepackage[margin=0.5in]{geometry}
\usepackage{graphicx}
\usepackage{listings}

% \includegraphics[scale=1]{img.png}
% \begin{lstlisting} 
\title{5. Interrupts}
\begin{document}
\maketitle
\begin{large}\textbf{Forklar de forskellige interrupt typer}\end{large}\\
\line(1,0){510}
\begin{itemize}
	\item Hardware interrupts
	\begin{itemize}
		\item Internal interrupts genereret af interne enheder
		\item f.eks. UART interrupt p\aa\ RX buffer fuld
	\end{itemize}
	\item Software interrupts
	\begin{itemize}
		\item Event driven software(e.g. GUI)
		\item Exceptions genereret af CPU (MMU errors etc)
	\end{itemize}
	\item Level triggered - Interruptet er aktivt s\aa\ l\ae nge IRQ'en er aktiv
	\begin{itemize}
		\item ISR skal godkende IRQ kilden, for at lave IRQ linjen blive deaktiveret
	\end{itemize}
	\item Shared interrupts
	\begin{itemize}
		\item Flere enheder p\aa\ en enkel IRQ line
		\item ISR st\aa\r for at bestemme kilden
		\item Enheder/client har som regel sit eget IRQ register
	\end{itemize}
	\item Edge-triggered interrupts
	\begin{itemize}
		\item ISR bliver kun kaldet en gang
		\item Afh\ae ngig af arkitektur, en IRQ bliver kun set hvis ISR routinen ikke k\o rer
		\item God til korte IRQ der ikke beh\o ver ack
	\end{itemize}
\end{itemize}

\begin{large}\textbf{Hvad er de typiske opgaver for en ISR}\end{large}\\
\line(1,0){510}
\begin{itemize}
	\item Skal v\ae re hurtig
	\item M\aa\ ikke sove
	\item M\aa\ ikke allokere hukommelse
	\item "Acknowledge" irq-pending bit p\aa\ enheden
	\item V\ae kke enheden
	\item Hente data fra enheden
	\item V\ae kke sovende funktioner
	\item Returnere IRQ-HANDLED eller IRQ-NONE
\end{itemize}
\begin{lstlisting}
	irqreturn_t my_ISR(int irq, void *dev_id);
\end{lstlisting}

\begin{large}\textbf{Hvordan virker interrupt h\aa ndtering i Linux?}\end{large}\\
\line(1,0){510}
\begin{itemize}
	\item \textbf{For blocking IO:} Applikationen sp\o rger efter data p\aa\ devicet, og blockerer til data er tilg\ae ngelig(event)
	\item I user space benyttes blokerende fil adgang
	\item I kernel space bliver dette implementeret i driveren(read/write sover)
	\item En sovende process tages ud af scheduleren's run-queue
	\item \textbf{Interrupts er KUN tilg\ae ngelig i kernel space!}
\end{itemize}

\begin{itemize}
	\item Top half - ISR
	\item Bottom half - workqueue, tasklet
\end{itemize}
	
\begin{large}\textbf{Hvilke funktioner skal vi benytte is en driver for at underst\o tte interrupts?}\end{large}\\
\line(1,0){510}
\textit{Request og free skal benyttes i open/close ikke init/exit!}
\begin{lstlisting}
	int request_irq(irqLine, ISR function, flags, devicename, device ID);
	void free_irq(irqLine, device ID);
	wait_event_interruptible(wait queue, flag);
	wake_up_interruptible(wait queue);
\end{lstlisting}
\begin{itemize}
	\item For at kunne finde hvilken IRQ line der skal requestes, kigges der i interrupt handler koden
	\item \textit{arch/arm/omap2/irq.c}
	\item gpio til irq metode
\end{itemize}

\begin{large}\textbf{Hvordan kan vi optimere vores ISR?}\end{large}\\
\line(1,0){510}


\end{document}
